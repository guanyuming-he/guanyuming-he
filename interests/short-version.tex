\documentclass[10pt]{article}

\usepackage{hyperref}
\usepackage{xurl}

\begin{document}
I am drawn to research at the intersection of secure-systems engineering and
applied cryptography, particularly those that protect people's digital rights,
such as privacy engineering and censorship circumvention. I'm from a place
where aggressive and cruel surveillance and censorship are everyday realities
\cite{internet.coup},
but many governments of the freer countries have been trying to take control of
cryptography and thus digital freedom of ordinary people for more than thirty
years, nonetheless \cite{anderson.freedom, review.crypto.war}\footnote{In fact,
	Prof.~Anderson's this section \cite{anderson.freedom} on his homepage was
	one major inspiration that set me on the path of cybersecurity; I admire
	his honesty to problems which I aim to approach my work with --- to
	confront problems as they are, to fight against injustice even when
inconvenient, and to maintain my integrity throughout my academic journey.} ,
despite strong expert oppositions \cite{keys.under.doormats,
bugs.in.our.pockets, chatcontrolchildprotection}. On top of my strong desire to
address these real world problems is my deep curiosity for foundational
reasonings, which will be discussed at the end of this statement, along with
why I think your research group is a good fit for me\footnote{If you are hooked
	after reading this short version and have time,
	check my more lengthy and general version of my statement of interests
	here:
\url{https://github.com/guanyuming-he/guanyuming-he/blob/main/interests/main.pdf}.}.

Security awards both attention to detail and high-level systematic view; its
subfields connect directly with both theoretical and practical areas that
fascinates me. This is because security is a weakest-link game, a
vulnerability in any aspect will ruin a system's security. Starting from
low-level details: beside the threat model, we actually implicitly assume
another model or host, that encapsulates the discussion. For
example, in isolation, the kernel is the host for process isolation, the
hypervisor that for VM isolation, and the physical world is the ultimate
host for all. As a result, any host vulnerability will result in new attacks,
a fact that is well exemplified by Meltdown, which breaks both the kernel and
hypervisor's isolation of memory, because the hardware, an outer host, has a
vulnerability \cite{meltdown}. Yet it touches more than a single logical flaw,
as it relies on the timing difference of cache accesses to extract information,
relating to the Spectre attacks family \cite{spectre}, which, unlike Meltdown,
relies on not a strict flaw but arguably a feature, implemented without
comprehensive knowledge of its side effects in outer hosts. 
However, there's little hope to understand and model the universe thoroughly
and accurately in one's life-time, and one can expect endless discoveries of
such side-channel attacks \cite{side.channel.1, side.channel.2,
side.channel.3}. Is it futile trying to cope with every possible one of them? I
view this as an everlasting dynamics within the cybersecurity community, where
people continuously push the boundary of our understanding, and this
consequently expands the requirements for secures systems afterwards. For me,
continuously challenging the current understanding of secure systems is both a
requirement and a philosophical pursuit to understand the universe better in an
application-oriented way.

In the opposite direction of going deeper, one also needs to go higher to
have a systematic view of systems. In particular, the current networking
infrastructures are messy and conflicting, in my opinion. Whereas the
fundamental problem of the infeasibility of establishing unique links between
every two computers demands sharing of links and thus welcomes centralization,
the problem of scaling simple network design to the global level resulted in
the complex and somewhat decentralized design of the Internet protocol, which
delegates most routing tasks and policies to subnets. This current reality
happens to be just right for localized censorship and surveillance: the
centralization of local networks enable a censor to effectively implement
measures, yet the decentralization of them at the global level prevents it from
screwing everyone else on the Internet and being strongly opposed. Although new
protocols (e.g. QUIC \cite{quic}) and even dedicated networks for digital
rights (e.g. TOR \cite{onion.routing, tor}, Matrix \cite{matrix}) have
continuously emerged, their unfortunate reliance on the Internet
infrastructure makes them highly vulnerable to sabotage and blockage. Is it
ever possible to achieve more security goals even on top of such highly
controlled infrastructures? And can we build a better infrastructure that comes
with a high incentive for its adoption, because the Internet is hard to change
\cite{why.internet.just.works}? These are two core problems I aim to answer in
my research in secure systems.

Being able to achieve security goals despite very powerful adversaries, such as
establishing a secret key known only by the two parties over an insecure
channel, is what modern cryptography charms me the most. Since 1976 when Diffie
and Hellman built upon Merkel's idea to open the door to public cryptography
\cite{new.directions.crypto}, a number of such ``paradoxical'' methods have
been discovered, like zero-knowledge proofs \cite{zero.knowledge,
zero.knowledge.np} and homomorphic encryptions \cite{first.full.homo}. I pay
most attention to two implications of them. One is that they update our
understanding of certian theoretical lower bounds of how much security one can
achieve in the face of adversaries only limited by feasible computational
bounds. The other is how they help minimize the trust required to perform
global-level colloboration (e.g. Bitcoin \cite{bitcoin}) that would typically
require a mutually trusted central authority to organize --- now it only
requires trust in cryptographic assumptions such as some problems are hard.
Moreover, I would expand this to regard crypto as a potential mitigation for
some serious problems in our current democracies: fragmentation and erosion of
trust. When equipped with cryptogrpahic protections such as encryption, a
certain amount of power
and freedom are given back to the people --- governments need resources to
crack one's system, be it burning a vulnerability, social engineering, or
coercing, when they can't simply overcome the crypto barrier that
\emph{passively} unites people's power together, in the sense that massive
invasion of digital rights requires the governments to multiply the efforts
requires on a single person by the number of those who are protected by crypto
constructs\footnote{This simplifies the issue by ignoring vulnerabilities,
	phishing techniques, etc., that can target many at once, but these are
usually dealt with quickly once burned.}. In this way, cryptography does
something more profound, uniting people's power passively, when prospects of
active collaboration are damaged by trust issues \cite{trust.book.fukuyama,
how.democracies.die}. Applying existing cryptographic tools, or discovering new
ones (once I'm able to), to passively unite peoples' power against massive
digital abuse, is my real world drive to research in cryptography. Besides,
cryptography also connects with so many theoretical areas, particularly
complexity theory, number theory, and group theory, which serves a good
entrance to my future academic journey.

My PhD goal is to address state-of-the-art problems in protecting digital
rights, with techniques involving secure systems design, analysis, or applied
cryptography. My long term goal in the academia is to become a professor and
start exploring deeper philosophical questions at the foundation such as
metamathematics when my ability permits me, while also focusing on security
issues. I envisage a future where I advance knowledge, spread awareness of
social problems, and also satisfy my deep curiosity in understanding humanity,
logic, and life better in fundational ways.


\bibliographystyle{acm}
\bibliography{main.bib}

\end{document}
