\part{Additional information}

\section{My interest in teaching and assistantship}
Although I have not had any experience as a true teaching assistant, I have
always been attracted to the prospect of teaching others with my knowledge. In
fact, when I am learning something exciting, I will often imagine presenting it
to others and be excited about it if I believe I found a nice way to do that.
(You know, I believe Section~\ref{sec.low.level} and \ref{sec.crypto} are such
examples. Why not take a look if you have not already?)
Similarly, when I am writing something that I want others to see, like this
document, I often take the position of the reader and wonder if she will be
touched by my writing and if she can get my ideas clearly.

To talk more specifically, I am particularly fond of Prof.~Winston's teaching
style, which I discovered when I started self-learning his 6.034 at MIT
\cite{winston.ai}. One important thing I learned from him is that he
demonstrated passion consistently during his teaching, whereas some other
professors I experienced have chosen to present things more objectively.  That
comparison made a huge difference in my perspective --- I feel more engaged and
inspired by his teaching style. I also went on to listen to his \emph{How to
Speak} lecture \cite{winston.htspk} after finishing learning his course.

\section{My career plan}
I plan to enter the academia after my PhD and become a professor somewhere
there is enough freedom of academic expression. The condition is important
because I come from a place where politics and ideology alignment is the
priority, and I see firsthand how damaging that is.

My plan to become a professor is a result of these three things:
\begin{enumerate}
\item My strong academic intersets and my passion in teaching as described
	early.
\item My wish to spread knowledge, advancing my own as well as the others'
	understanding of the subjects I study.
\item Finally, my personal commitment to live and work ethically in a way that
	does not require freedom to come at someone else’s expense. It may be
	easier to gain freedom by aligning with exploitative structures, but I
	cannot accept that path. I hope to take a different approach: to find and
	strengthen spaces within academia where knowledge can be shared openly,
	collaboration can be fair, and students can pursue ideas without fear or
	compromise. Though I recognize that academia is not perfect and  can be
	exploitative (to a less degree), I believe it still offers the best
	opportunity to practice and promote a more just environment in which I work
	and devote my passion.
\end{enumerate}
