\documentclass[10pt]{article}

\usepackage{amsmath, amsthm, amssymb}
% See amsthdoc p.~3
\newtheorem*{CT}{Cantor's Theorem}

\usepackage{hyperref}
\usepackage{xurl}

% Only include additional info to compile if I am giving the document
% as a personal statement in my PhD applications.
\includeonly{additional}

\begin{document}
\tableofcontents

\section*{About this document}
This document gives a comprehensive discussion of my current academic
interests. They are categorized into broad fields as sections, and each
subsection, describing my interest for a subfield of it, will generally give my
motivation for studying that field and usually end with a current open problem
that I aim to solve during my research in that subfield.

If you are a professor, I might have sent the whole, a part, or a modified
version of this document to you, as my statement of interest. Generally, you
only need to read the subfields that I want to be working in with you; see the
above table of contents. If you have time, however, I still recommend a linear
read through the document to see the whole picture of my ambition and passion.
Finally, to get the latest version of the document, go to
\url{https://github.com/guanyuming-he/guanyuming-he/blob/main/interests/main.pdf}.

\part{My academic interests}

\section[Security]{Security: Introduction into my interests}
I believe security is a good entrance into my current interests. It awards both
attention to detail and high-level systematic view; its subfields connects
directly with both theoretical and practical areas that fascinates me, and it
offers effective tools to solve the imminent problems I face in real life. This
section will take such a flow that the subfields of my interest will be first
led to from real world problems and motivation, and then be explored with my
more academical and philosophical pursuits.

The term \emph{security} in computer science generally carries the sense of
achieving \emph{security goals} through \emph{mechanisms or properties} of a
system, despite the presence of adversaries in a \emph{threat model}. The
threat model has to be carefully chosen and reviewed; there would be no
security against an omnipotent adversary.

Unlike in math, one cannot arbitrarily wiggle the threat model to one's desire.
On the contrary, one has little control to how one's digital rights are
treated, which are increasingly threatened (e.g. \cite{eu.digital.1,
eu.digital.2, internet.shutdown.2024}). A more prominent example is the
continuous push for exceptional access to people's data and communication in
various forms by governments\footnote{ For a recent instance, EU revived the
\href{
https://eur-lex.europa.eu/legal-content/EN/TXT/?uri=COM\%3A2022\%3A209\%3AFIN}
{Chat Control} proposal in 2025. See \url{
https://eutechloop.com/time-is-running-chat-control/}.}, ignoring expert
opinions against such measures \cite{keys.under.doormats, bugs.in.our.pockets,
chatcontrolchildprotection}. Prof.~Anderson has frankly described the fight for
the control of cryptography (thus a large portion of peoples' digital rights)
between governments and all who try to fight back with the term ``(thirty years
of) crypto war'' \cite{anderson.freedom}. In this section, you will see a
similar tone to Prof.~Anderson's honesty which I admire --- I aim to approach
my work with the same honesty --- to confront problems as they are, to fight
against injustice even when inconvenient, and to maintain my integrity
throughout my academic journey.  In fact, Prof.~Anderson's section
\emph{Privacy and freedom issues} \cite{anderson.freedom} on his homepage was
one major inspiration that set me on the path of cybersecurity, but which I
will not elaborate in the document.

Moreover, I unfortunately have lived in a place where
peoples' digital rights are much more aggressively invaded, particularly in the
form of systematic, comprehensive, and far-reaching censorship and surveillance
\cite{internet.coup} \cite[Sect.~5]{chall.censor.circum}.  These threats
present one of the major real world problems I aim to address by my research in
security and my main realistic motivation. Through the document, you will see
my philosophical motivations as well.

\subsection[Secure Systems]{
Secure Systems: Understanding the mechanisms} 
\label{sec.secure.systems}

\subsubsection{Low-level system details} \label{sec.low.level}
Beside the threat model, when we talk about security, we also implicitly assume
another model, environment, or host, that encapsulates the problem. For
instance, in the context of isolation, 
\begin{itemize}
	\item The host that encapsulates process isolation is the operating system
		kernel.
	\item The host that encapsulates virtual machine isolation is the
		hypervisor, often including hardware support.
	\item The host that encapsulates air-gapped machine isolation is the
		physical world, or more abstractly, the physical laws.
\end{itemize} 
These hosts form a hierarchy. For instance, processes inside an OS may run
inside a virtual machine, on hardware, within the physical world.

As a result, a security researcher has to understand these hosts, each to a
different degree, depending on how secure one wants one's systems to be. Even if
a system is formally verified to be secure within a host, it can still be
attacked from interactions with an outer host. That is well examplified by
the Meltdown attack, which breaks both kernel and hypervisor's isolation of
memory, because an outer host, the hardware, has a vulnerability
\cite{meltdown}. In fact, Meltdown touches more than logical flaws in hardware;
it relies on the timing difference of cache accesses to extract information
from it \cite{flush.reload}, one that directly connects with the physical
world. At this point, I feel that I should also mention the Spectre attacks
family \cite{spectre}, which, unlike Meltdown, does not exploit a clear logic
bug or fault in the hardware. Instead, it work by observing the side effects
of various implementations of speculative execution, which is arguably not a
bug but simply a feature implemented without caution to how information may be
leaked via the outer hosts.

Indeed, there is little hope to understand and model the universe thoroughly
and accurately in one's life-time, and one can expect endless discoveries of
how an otherwise secure system can leak information in unexpected ways, such as
power and radiation via physical effects, which opens to a large range of
attacks \cite{side.channel.1, side.channel.2, side.channel.3, flush.reload},
collectively defined as side-channel attacks.  Then, one must ask, is it futile
trying to cope with every such possible attack? My opinion is that it depends
on the current understanding of the whole cybersecurity community. One novel
attack just discovered may be increasingly understood and gain popularity by
the whole community and thus gradually become a necessary part every secure
system designer should take into account. I view this as an everlasting
dynamics where people continuously push the boundary of our understanding, and
consequently expanding the requirements for all in the community.

Therefore, a good understanding of the most common and well-understood hosts,
the operating systems and the hardware (including the architecture), will be
essential for me as a security researcher. Additionally, figuring out how to
perform clever hacks (as in the hacking culture of MIT, not the mainstream
meaning of cracking) based on knowledge of details of a system gives a great
sense of achievement to me. Moreover, this continuous expansion of knowledge
frontier carries a special implication in security. Whereas a mathematician is
perfectly fine to specialize in one direction without knowing every details in
other fields, a computer system can never be secure if only designed with one
perspective. For now, a good system is usually designed by collaboration; and
it will be a very interesting challenge to figure out how one single researcher
can be empowered to do this task better; could the LLMs be a potential
assistant who fills the knowledge gaps? Finally, this is one major place where
cybersecurity connects directly with a broad range of other fields, including
software engineering, systems architecture, networking, and physics.  These
together contribute to my interest in the low-level details of secure systems.

\subsubsection{A high-level view of the Internet} \label{sec.high.level}
A high-level understanding of how the digital infrastructures work is essential
to understand what enables the threats to digital rights on top of them. The
current networking infrastructure, in my opinion, has conflicting properties.
On one hand, the fundamental problem of the impossibility to link every two
computers requires sharing of links, a solution that welcomes centralization.
On the other hand, the poor scalability of simple sharing schemes of a link
(usually via a switch) necessitates a better scheme to extend a small network
globally. To achieve this, the Internet relies heavily on delegation and
distributed structures, such as topological divisions of its address space and
delegation of most routing responsibilities to each individual networks (e.g.\
ISPs). Its BGP, opearing in between them, is mainly concerned exchanging
reachability information, whereas its IGPs handle routing paths and allow
different networks to implement different routing policies.

Therefore, the Internet has become a mixture of centralization and
decentralization, where each end node is managed by an ISP network, yet no
single ISP runs the whole Internet. Perhaps surprisingly, I found this hybrid
structure more optimal for localized sabotage at the state level, creating
``sub-Internet''s, each of which is crippled at a different level. 

Unfortunately, the lower layers of the Internet have proven to have a great
inertia for change.  Handley gave a nice discussion on it in 2006
\cite{why.internet.just.works}, and the trend he described has mostly been the
same since then: ``\emph{the core Internet protocols have not changed
significantly in more than a decade, in spite of exponential growth in the
number of Internet users and the speed of the fastest links.}''
\cite{why.internet.just.works}. On the other hand, the protocols in the higher
layers evolved to a much greater degree. As the transport layer welcomes
QUIC\cite{quic}, the application layer has accumulated an enormous amount of
innovation and progress, in particular, onion routing and
Tor\cite{onion.routing, tor}, Bitcoin\cite{bitcoin}, VPN
protocols\cite{openvpn, wireguard} and decentralized instant chat\cite{matrix,
tox}, that fight for digital rights and/or promote decentralization. Yet one
must not overlook the crypto-constructs that lay the foundation for all of
them, which I discuss in detail in Section~\ref{sec.crypto}.

Unfortunately again, because all of these are built on the Internet, a
state-level censor could easily abuse its local authority to target them. A few
regimes are notorious to have blocked a vast amount of them to various
extents, employing complicated passive analysis and active probing techniques
\cite{censor.block.1, censor.block.2,censor.block.3, censor.block.4,
censor.block.5,censor.block.6}.
Although a state would need to consider the collatoral damage, a totalitarian
regime would not hesitate to block an entire protocol before it can figure out
how to block it selectively \cite{selective.block.1, censor.block.4}. 
Apart from state-level actors, because commerical local ISPs additionally
have an incentive in income and profit, not only do they perform surveillance
and censorship like state-actors \cite{isp.statelike.actions.1,
isp.statelike.actions.2}, they also implement unjust policies easily with their
local control of the infrastructure, like unfairly limiting the use of certain
P2P protocols \cite{isp.block.p2p.1, isp.block.p2p.2, isp.block.p2p.3,
isp.statelike.actions.1}.  Although ISPs argue that these P2P protocols can
consume too much bandwidth, the other side of the story is that ISPs often
oversubscribe and fraudulently advertise the bandwidth of Internet service they
provide \cite{isp.oversubscribe.1, isp.oversubscribe.2}. This essentially is a
probabilistic exploit on its customers --- when almost all of the users happen
to use the maximal bandwidth the ISP sells to them, congestion and throttling
occur, and the users, not the ISP, ultimately pay the price --- this is also
the same kind of injustice imposed by airlines who oversell tickets.

The question of how to build protocols and systems that preserve one's rights
on top of the Internet that powerful adversaries control, fighting against
surveillance, censorship, and overall other unfair practices which the
infrastructures of the current networks happen to enable, is central to my
interest in the high-level design of secure distributed systems.


\subsection[Data-driven analysis]{
Data-driven analysis: Lights through artifical clouds}
Despite my intention to understand systems, many of them are not open for
analysis, for various reasons. In particular, the aforementioned systems for
digital surveillance and censorship are not only proprietary but often state
secrets. However, just as no material can perfectly insulate against heat in
physics, no system can perfectly isolate information; some of it inevitably
leaks through side channels (see Section~\ref{sec.low.level}). 

By carefully observing their behaviors and probing their responses under
controlled conditions, we can gain insight. Combined with data science
techniques, these observations reveal hidden structures and enable us to draw
meaningful conclusions.

The process usually involves 1) raising questions and claims about some
properties of a target system 2) conducting experiments (often involving
purchased servers within such as Aliyun) 3) analyzing data 4) answering
questions and verifying claims. As a simple example, Sheffey et al.\ very
recently studied the IP addresses injected by the GFW censorship system. By
first finding injected IPs and then probing them within the GFW, they found
three categories of such IPs \cite{gfw.injected.ip}. 

Thanks to continuous work from both academic scholars and dedicated
organizations, (e.g.\ \cite{data.analysis.1, data.analysis.2, censor.block.6}),
I and a small set of others who suffer from such systems and who are fortunate
enough to know these works, could know what we are facing everyday better.
Conversely, I strongly hope to contribute to the free-side of the arms race,
making the free Internet reachable to everyone.

\subsection[LLMs and security]{LLMs and security: A timely matter}
The contemporary popularity of LLMs in both the academia and the general public
is so obvious that I don't need to cite to support it. One particular public
usage of them is for information retreival \cite{info.ret.llms.1,
info.ret.llms.2}, which has many security
implications. Among them, my interests are in: 1) how to preserve the privacy
of all who are involved in the training data. 2) how to circumvent LLM
censorship. With all these said, LLMs are not of as much theoretical interest
to me as any of the previous or following subjects; I study this matter mainly
because of its wide usage and thus its impact.

Regrettably, I do not have enough knowledge to write meaningful things
about 1). But I do for 2). At first sight, it would seem to be a tricky
problem, because even to this day, non-trivial neural networks are still mostly
blackboxes whose connections we know little about. However, there are actually
surprisingly many ways to censor such a blackbox, combining censorship on
input, output, tokens, weights (to limited extent), and also using specific
fine-tuning to train models to reject undesired contents

Where there is a way to censor, there is a way to defend. First, we can poison
the training data \cite{data.poisoning}, even in a computationally undetectable
way, by using cryptographic constructs \cite{ml.undetectable.backdoor,
ml.undetectable.backdoor.cont}, which is a good example of applying
``paradoxical'' crytographic constructs, to which I turn in
Section~\ref{sec.crypto}. Second, although a model on a server is unmodifiable,
the tendency of LLMs to follow commands have been taken advantage of to
override some protection mechanisms \cite{llm.jail.break}. Finally, one can
always request a LLM to speak in some encoding\footnote{As the time of writing,
it is completely doable to role-play with ChatGPT by declaring that I am a
cryptographic professor who wants to set assignments for students to
practice some cipher, say, the Caesar ciper, and that I want ChatGPT to
predict student replies. Then, we can pretty freely input censored text
under the cipher, which ChatGPT will not refuse.  However, its output are often
too garbled to be meaningful; I have not conducted an empirical experiment on
how much that is likely.}.  

Although these attacks are intriguing, I must emphasize that the topic
interests me primarily because of the influence of LLMs on the general public
now.  Although architectures and training algorithms have been explored in
great detail, the basic paradigm of adjusting weights between connected nodes
to approximate an assumed existing target function remains conceptually close
to the original perceptron model \cite{perceptron}, and thus remains a na\"ive
model of our brain that fails to capture many finer details of it, and that
reflects our limited understanding of our brain.  Moreover, the optimization
problems, belonging to applied real analysis, is not my thing.  Thus, I have
overall limited interest in the theories of LLMs, unless there emerges the next
big idea in neural network based machine learning.

On the other hand, I do have interests in the application of LLMs. I generally
agree with LeCun, Bengio, and Hinton in that the current success of deep neural
networks is largely due to their ability to absorb a massive amount of data
(especially raw data) with little engineering \cite{2015.deep.learning}.  As
such, even with their logical inaccuracy and other problems, they still have
great potential to speed up our works. Consider a massive pump of water that
contains some waste (c.f.\ hallucination); so long as we could correctly filter
the waste out, we can still get uncontaminated water at a high rate. As an
interesting example, in Section~\ref{sec.speed.formal} I noted how LLMs can
potentially speed up formalization, with proof assistants as the filters. 

\subsection[Cryptography]{Cryptography: A mathematical
savior?\protect\footnotemark}
\footnotetext{I use the question mark, because cryptography cannot be, at least
not a savior alone, if other problems are not solved. See the discussion at the
end of this subsection.} \label{sec.crypto}

\subsubsection{The potential of ``paradoxical'' modern crypto constructs}
Security is often described as an arms race --- that who controls more
resources tends to discover more vulnerabilities, devise more attacks, and
harden their systems more. Where is hope, then, when most self-censor and give
up fighting the regime \cite{self.censor.1, self.censor.2, self.censor.3}  and
even fewer of the fighters have sufficient skills to join in the security war
\cite{defenders.lack.skills.1, defenders.lack.skills.2}?

I believe one partial solution to what might sound like a power struggle lies
within (modern) cryptography. For thousands of years since the use of the
earliest symmetric encryption methods like the Caesar cipher, the ability to
securely communicate through an insecure channel had still been mostly limited
to the privileged who could afford persistent access to physical secure
channels to exchange the keys and reliable safeguarding of the keys.  A
stunning turning point was found in what was widely regarded as the beginning
of modern cryptography, Diffie and Hellman's \emph{New Directions in
Cryptography} \cite{new.directions.crypto}, where they gave a practical
mathematical procedure to Merkle's original idea for establishing a key known
only to both parties over an insecure channel.

At the time, the idea seemed so paradoxical that Merkel himself faced 
rejections by his then professor Hoffman and the
CACM \cite{merkle.rejection}:
\begin{quotation}
	``I am sorry to have to inform you that the paper is not in the main stream
	of present cryptography thinking and I would not recommend that it be
	publioned in the Communications of the ACM.''

	``Experience shows that it is extremely dangerous to transmit key
	information in the clear.''\cite{merkle.rejection}
\end{quotation}
Even Shannon shared that assumption: ``\emph{The key must be transmitted by
non-interceptible means from transmitting to receiving points}''
\cite[p.~670]{shannon.theory.secrecy}.

Such ``paradoxical'' ideas of modern cryptography are what attract me most to
it, because the more paradoxical such an idea is, the more power it returns
individuals to preserve their rights against overwhelmingly strong
adversaries\footnote{Albeit unfortunately they all rely on some assumptions, to
which I return in Section~\ref{sec.crypto.limit}.}.  Thus, such ideas become
the enabler that underlies those systems mentioned in
section~\ref{sec.high.level}: TOR, Bitcoin, Matrix, etc.\footnote{Yes, the
powerful adversaries can of course use them; but the main point is cryptography
enables what was not possible for the weak before.}, and the catalyst that
leads to the massive adoption of society advancements such as e-commerce.
Remarkably, in less than 50 years since 1976, we already have a number of such
discoveries in addition, and here's a list of some of them.
\begin{description}
\item[Pseudorandom functions] The kind of deterministic algorithms
	whose outputs look like that of a random oracle, by block ciphers like
	AES\cite{aes} or other ways \cite{pseudo.rand.cons.2}, has a strong link
	with various other essential constructs like one-way functions.

\item[Zero-knowledge proofs] By interactively leveraging challenges that only a
	true prover could easily solve, zero-knowledge proof \cite{zero.knowledge}
	enables one to verify that the prover knows a witness of a problem in NP
	\cite{zero.knowledge.np} without revealing the witness.

\item[Homomorphic encryptions] Doing computation on encrypted data has been a
	very desirable property for many years. Although many earliest public-key
	encryption schemes, such as RSA \cite{rsa} and ElGamal \cite{elgamal},
	already natively supported limited homomorphism like modular
	multiplication by their design, the first full scheme that allowed
	arbitrary computation on encrypted data was only proposed in 2009
	\cite{first.full.homo}.
\end{description}

These discoveries have many applications and implications, two of which I pay
most attention to. Theoretically, they update our understanding of certain
theoretical lower bounds of how much security one can achieve in the face of
strong adversaries, no matter the power difference between them (provide the
adversary is still bounded within feasible computational limits, of course).
Socially, they help minimize the trust required to perform global-level
of collaboration that involves people from vastly different backgrounds and
beliefs; such a collaboration that would typically require a mutually trusted
central authority to organize, now only requires the participants to trust a
few foundamental assumptions of cryptography.

Personally, I would go a step further. I believe modern cryptography offers not
only technical tools, but a possible mitigation for one of the big
challenges our democracies face today: fragmentation, polarization, and the
erosion of shared trust \cite{trust.book.fukuyama, how.democracies.die,
cultural.backlash}. By
design, cryptography redistributes power to a certain degree --- governments
need resources to crack one's system, be it burning a vulnerability, social
engineering, or coercing, when they can't simply overcome the crypto barrier
that unites people's power together, in the sense that massive
invasion of digital rights requires the governments to multiply the efforts
requires on a single person\footnote{This simplifies the issue by ignoring
vulnerabilities, phishing techniques, etc., that can target many at once, but
these are usually dealt with quickly once burned.}. In this way, cryptography
does something more profound, uniting people's power passively, when prospects
of active collaboration are damaged by trust issues.  Fortunately, as many
modern infrastructures already deploy encryption and other crypto schemes by
default, this ``passive solidarity'' is not a dream but a reachable reality,
one that shows how mathematics can serve as a partial safeguard for democracy
in an increasingly divided world.

\subsubsection{Limitations and users' responsibilities}
\label{sec.crypto.limit}
Having all hopes on passive constructs to save people's active lives is
unrealistic. As a direct problem, crypto constructs all have assumptions,
particularly that some problems are hard.  Although P vs.\ NP seems rather
remote to one's normal life,  more serious limitations emerge when applying
cryptography: that is when a theoretical model fails to capture all the danger
that can break the assumptions. As a very practical example, I have been
devoted to using OpenPGP since I know it. One central assumption of it is that
one's private key is never compromised. This rather simple assumption on paper
brings a lot more serious problems when actually in use.
\begin{enumerate}
\item A OpenPGP secret is usually used for at least a few years. Although the
	risk of a system compromise at one time is low, it accumulates as time
	passes.  Even as someone aware of security issues and practices, I have no
	confidence in that there will be no breach (more strictly speaking,
	opportunities of breaches) to my system once a few years. 
\item As such, people have used many practices to safeguard the secret, such as
	writing it down physically and storing it in a vault, or use a more
	convenient media such as a hardware key. But if we want a key to be useful
	at all, we need to use it, and each use increases the risk of compromise;
	even if a hardware key prevents the secret from being extracted, a
	compromised system can still trick the key into performing operations,
	combined with some authentication breach. 
\item An arguably safer way that reduces interaction with the secret is to have
	a master secret stored in isolation such as on a hardware key, and then
	create subkeys that are signed by the master secret and used daily. The
	master secret is only used to manage subkeys.
\item I personally use the above scheme, but a subkey's compromise can still be
	fatal in many cases. One specific problem is the lack of forward secrecy
	when a subkey is used for encryption. What's worse, one who only wants
	to snoop on my communication can stay low-profile, and I may never get to
	know a subkey is compromised before its expiration date.
\end{enumerate}
Thus, although those ``paradoxical'' properties of crypto constructs are
strong and many remain sound for many years, real world details really affect,
and in many cases, decide how secure one's systems are in daily usage. 

Since such issues are faced even by security researchers everyday, the general
public clearly will suffer more from them. Anderson reported that cryptosystems
in retail banking failed mostly because of ``implementation errors and
management failures,'' instead of cryptoanalysis problems, in 1993
\cite{anderson.why.cryptosys.fail}; and that observation remained mostly the
same, 25 years later \cite{why.cryptosys.fail.revisit}. From another domain,
Adams and Sasse surveyed users of password systems and argued that the
then such systems lacked a user-centric design, also identifying a ``vicious
circle'' where security policy makers and users lack a mutual understanding
\cite{users.are.not.the.enemy}. 

One direction to address this problem is to develop more easy-to-use, or in
another phrase, more fool-proof systems that hide complexity from users
\cite{secure.systems.people.can.use}. often by combining crypto constructs in
more complicated ways to patch known big issues.  For instance, the forward
secrecy problem is patched by most instant-messaging systems with real-time
session key (ephemeral key) establishments.  In particular, there is a recent
trend to make systems ``secure by default'' \cite{secure.by.default}. That
raises the important question: can security can really be achieved
automatically and passively by default?

I argue that it cannot --- there are essential complexities of security that
should not be hidden. The use of ephemeral keys decouples encryption and
decryption from the master secret, thereby protecting it from exposure through
routine operations; but what stops a careless user from handing out freely what
we try so hard to protect? And doesn't building more fool-proof systems risk
spoiling users and decreasing the overall awareness of the responsibility that
comes naturally when rights-preserving systems return power to users
\cite{usable.security.contradiction, rid.of.usability.sec.tradeoff}?  Thus, I
contend that we should instead build systems that encourages and directs
users to actively engage with their own security. Consistent with my view,
in the afore-cited work of Adams and Sasse, they did not advocate for providing
for the users at all cost; instead, they suggested that many users are aware of
security but need to be informed of the circumstances and be trained
\cite{users.are.not.the.enemy}.  Similarly, both Anderson and K\o ien,
separated by 25 years, arrived at the same conclusion that there is a
deficiency in security competence and it is imperative to educate and spread
awareness more about security \cite{anderson.why.cryptosys.fail,
why.cryptosys.fail.revisit}. Finally, I find the underlying reasoning very
similar to that of why freedom is not free, to which I return in
Section~\ref{sec.free.software}.

To end this section, cryptography is the enabler of and essential for the hope
of the digitally abused to break free from the powerful adversaries, but on the
other hand, it is not a savior for those who lack basic understanding of
security. To apply cryptography in an engaging way that automatically incites
users to learn more about security, is a central research direction I want to
explore. Besides, because of its deep connection with complexity theory, number
theory, and other theoretical fields, it also serves as a good preparation
before entering these realms in the later stages of my academic career.

\section[Software engineering]{
Software engineering: My earliest and continuing passion}
\subsection{Software construction}
The seed for my passion in questioning how things should be built and be amazed
at how complex things interact with each other when assembled together, was
there in me long before I had an idea of what research or the academia were. In
high school, I was obsessed with creating my own video games, what I hoped
would be my sanctuary from my then miserable life. My unsuspectingly ambitious
goal of starting from building a game engine unsurprisingly failed, but it did
introduce me to the book \emph{Game Engine Architecture}
\cite{game.engine.arch}, which isn't academically significant, but its contents
made me realize that I was more into how things work in the engine such as how
threads coordinate with each other than into actually making a game.

This interest of figuring out how a complex system work and how to build such
one, leads me into learning various software engineering principles, in
particular, those of Liskov \cite{liskov.adt, liskov.subtype}, learning them
deeper when offered by my undergraduate courses, and applying some of them in
all of my big projects, since I got to know them. 

With all these principles, engineering of large software systems still remains
a big challenge, particularly in interface design, separation of components,
and all that we still have no fixed rules to follow. In other words, large
systems engineering largely remains an art instead of a science
\cite{no.silver.bullet, prospect.eng.discipline.software}. Could it become a
science one day, governed by well-establioned rules that guide us toward
optimal software systems in all or at least most situations? That is an
intriguing question I aspire to explore while continuing to construct software
in the following years.

\subsection{Formal methods/verification}
Many software engineering principles are concerned with mistakes made by
humans. Not only do we try to minimize bugs from programs, we also need to
reduce them from the specification or requirements of the software.  Therefore,
formalizing the specification and verifying that the software built indeed
satisfies it could be seen as an ultimate way to eliminate bugs
\cite{hoare.axiomatic.prog}. 

\subsubsection{The speed problem of formalization} \label{sec.speed.formal}
Perhaps unsurprisingly, the industry has a low usage of fully verified software
\cite{formal.methods.underuse}, since the business dynamics changes rapidly,
and approaches that allow quick iteration of software, such as agile
development, are much more preferred \cite{agile.se}. However, even in the
academia, formal verification is not frequent, outside of critical areas. One
central reason could be that formalization of software is simply too
time-consuming \cite{formal.methods.underuse}; as a price of having reduced
software logic into primitive steps, much more of such steps are required.

That might seem to be an essential difficulty preventing the wide-adoption of
formal methods in software engineering, but another way of dealing with the
problem is increasing the formalization speed. One approach is to formally build
more advanced and verified steps out of primitive steps and define them in
\emph{proof assistants}. For example, Coq (Rocq) allows one to build custom
tactics and use a meta-language such as Ltac to manipulate tactics at a higher
level \cite{coq.manual}. Furthermore, LLMs, the current hot topic, have the
capacity to output seemingly correct formalization at high speed
\cite{llm.gen.proof}.  Combined with proof assistants, which can automatically
verify the correctness of the output, we could have the potential to greatly
speed up formalization of not only software, but also of informal mathematical
proofs found everywhere in mathematical texts, if the product of the correct
rate and the output speed of LLMs could surpass that of ours.

\subsubsection{How formalization may change collaboration}
One interesting application of proof assistants is Massot's \texttt{Blueprint}
\cite{massot.blueprint}. It is mostly a dependent graph that links each piece
(lemma) of a big proof as its nodes, which one can click on to go to its
rendered \LaTeX, \LaTeX\ source, and Lean (a proof assistant) source. Depending
on a node's formalization status, it's displayed differently in the graph
\cite{tao.blueprint.post}. Beside being used in various math projects (see its
README), it also has the potential to support building interactive math
textbooks for students.

In another direction, the deep trust that was previously required for each
collaborator's math skills, can now be reduced by automatic verification with
proof assistants. Thus, whereas a research-level math project used to be done
by a few professional mathematicians only, such one now can be instead
conducted this way: 1) the few expert lead mathematicians design a whole picture
of the theorem to prove and divide it into pieces 2) people, even those who are
not professional mathematicians, can claim pieces that they think they can
solve. 3) when they give solutions, it's the proof assistants' job to verify
them. As a proof of concept, Tao has led a pilot project conducted such way
\cite{tao.pilot.project}.

While most software engineering projects require far less trust than a
math project does, projects of engineering critical secure systems could
still demand a high-level of trust and thus benefit from proof assistants in
the same way as math projects above. In particular, could the development of a
fully-verified and quite complete (in the sense of containing drivers, file
systems, and other components, unlike a microkernel) OS kernel be possible, if
collaborated this way? That would be an interesting question to answer in my
future research.

\subsection[Free software, free society]{
Free software, free society\protect\footnotemark}
\footnotetext{The title here is the book title of the essays collection of
Dr.~Richard Stallman \cite{stallman.essays}.}
\label{sec.free.software}

As our lives are more digitalized, digital computers are playing an
increasingly important role in them. Consequently, one who controls the
computers will have great power over our lives. But do we control the computers
(in the general sense, including mobile phones, smart devices, etc.) that we
purchased and own? By the very definition, software controls our computers.
Then, the question becomes, do we control the software that runs on our
computers?

Unfortunately, the answer is no for proprietary software. Stallman
has a number of nice essays on this topic \cite{stallman.essays}, and I would
just use a simple analogy here: using proprietary software is like entrusting
one's money to another person, whose operations are opaque, whose decisions one
has little or no influence on, whose interests are mainly profit, and who
otherwise has no connection with one. Clearly, no sane person would choose to
do that.  The core problem of proprietary software is that via them the
developers impose such an unjust power imbalance over its users, which tends to
corrupt and lead to mistreatment of users, as history has repeatedly shown us.

By launching the free software movement, Stallman sought to restore power to
users by securing their essential freedoms \cite[Essay~1]{stallman.essays}. The
movement has achieved remarkable successes, yet today it still faces serious
obstacles. An prominent example is the current situation of open source,
perhaps surprisingly. Often associated with Raymond’s \emph{The Cathedral and
the Bazaar} \cite{cathedral.bazaar}, open source in fact emerged from within
the free software community as a deviation from its core philosophy. As
Stallman emphasizes, ``The term `open source' quickly became associated with
ideas and arguments based only on practical values, such as making or having
powerful, reliable software'' \cite[Essay~14]{stallman.essays}. By shifting
focus from freedom to utility, open source has obscured the very principles the
free software movement was founded to protect.

There are two significant problems of the current situation of open source. The
first is about the licenses. Some licenses only open the source code but
prevents unauthorized modification and sharing; some others grant total
access but fail to prevent the work from being stolen (i.e.\ creating a
proprietary fork from it). 

The second problem concerns the question whether freedom is passive and free.
Many users of free or open source software almost never inspect, modify, or
share the code \cite{open.source.contrib}, thereby remaining as dependent on
developers as users of proprietary systems. Their situation differs only in
that the developers of free software may act more ethically and the part of the
users who do exercise their freedoms can somewhat prevent the developers from
going rogue. Yet the open source movement's deliberate avoidance of freedom as
a guiding value undermines awareness of it within the broader community.  As
much as convenience is important, one should realize that the underlying free
software principles which encourages modification and sharing are the enabler
of the Bazaar model. The danger of discarding those principles is that it
creates an opportunity for corporate corruption and take over, undermining not
only freedom, but eventually the convenience the users seek at the beginning.
Today, corporate-led open-source projects, such as Microsoft's VS Code,
illustrate this dynamic. Although they are sustained by community
contributions, their governance, licensing, and branding remain firmly under
corporate control \cite{corporate.open.source, fake.open.source}.  The result
is software that strengthens corporate power while exploiting users (e.g.\
VSCode has built-in telemetry that is difficult to disable). 

That situation of open source, in my observation, is dangerously similar to the
rise of authoritarian populism discussed by Norris and Inglehard, who
understand populism as ``a style of rhetoric reflecting first-order principles
about who should rule, claiming that legitimate power rests with `the people'
not the elites. It remains silent about second-order principles, concerning
what should be done, what policies should be followed, what decisions should be
made'' \cite[p.~4]{cultural.backlash}. Similarly, I view most popular open
source projects as reflecting shallow principles of allowing everyone to
contribute to add more features, but ignoring deeper principles as to what
direction the software should take, which decisions to make when convenience
conflicts with freedom, and what are not allowed, even if they add more
resources to the development of the software. Although a difference is that
many populists reject the previous ``ruling elites'' whereas many open source
members may not strongly oppose free software, they all involve people who
commit to good principles such as democracy and open contribution only in the
form, but ignoring the ideas and actions behind.  Consequently, just as open
source projects are corrupted and taken over by corporations, Norris and
Inglehard observed that such shallow commitment to democracy in turn damages
democracy itself and welcomes authoritarian \cite{cultural.backlash}.

Before concluding the discussion, one should note that there are many pieces of
open source software which are very close to free software, which, although not
under a strict free software license, grants most or in some cases, all
freedoms to its users, and whose community do have people that care for these
freedoms. For these pieces of software, does how we call them matter; and can
we prefer the term ``open source'' over ``free software'' for popularity, since
free software is always open?  Stallman points out that it matters,
``\emph{because different words convey different ideas. While a free program by
any other name would give you the same freedom today, establishing freedom in a
lasting way depends above all on teaching people to value freedom}''
\cite[Essay~14]{stallman.essays}. 

Recall I argued in Section~\ref{sec.crypto} that we should review our current
trend of pursuing security by default, hiding all complexities in secure
systems, and I suggested to engage users in security more. I believe this is
essentially the same issue. Just as some previously free software opens the
door to corruption by using the name ``open source'' that implicitly teaches
people to prioritize convenience over freedom, the security community's focus
to build more fool-proof systems sets users more remote from the what security
is really about in its core, freedom from harm, power over one's life, and the
responsibility and complexity that inherently come with it.

I see this as a profound yet subtle undermining of the free software movement,
in which the core values of free software are silently lost amid the popular
chase for convenience. Stallman quitted his scentist career at MIT and became
an activist to dedicate his life to the free software movement. In the
meantime, can we contribute to it from the academia? Could we find ways to
build software that somehow helps users overcome the current trend of chasing
popularity and make the importance of freedom as well as security appear
pressing to them? And if we could, how can we ensure that users have incentives
to use our software instead of the mainstream ones, even when they are aligned
with the exploitative structures? These are not just important questions for
my future research in free and secure software to answer, they are also
essential for the society, if it is our goal as scientists to hope our
knowledge will end up serving people, instead of becoming the tools that
enslave them.

\section[Mathematics]{Mathematics: My purer philosophical pursuits}
\label{sec.math}
As much as I am driven by real-life problems to understand systems better, the
force behind my sometimes unusally deep dive into even more primitive
principles is my strong desire to understand the foundational reasonings. In
this section, I will describe how that has led me to explore the purer
mathematical areas and my current interests in them.

\subsection{Type theory and metamathematics}
In the manuals of various proof assistant languages, technical terms from type
theory such as universe, sort, and introduction \& elimination rules, are
frequently mentioned. For me who is driven to understand what's going on
behind, contact with type theory is my destiny. Furthermore, I ventured deeper
into its history and the philosophical doctrines it was born from.

I have thought of a few ways to present this section, and in the end I decided
that I will present it as a narrative of the basic history. If you share my
curiosity for foundational reasonings, then after seeing the history I
presented in my style, I believe you will also see my passion.  A good starting
point is probably Cantor's work on his theory of abstract sets, particularly
his ingenious abstraction of sets to cardinal numbers for their comparsion.  In
1638 Galileo noted that there is a bijection between the natural numbers and
their squares: $f : n \mapsto n^2$, raising questions about the principle that
the whole is greater than any of its parts. Cantor took this further to
systematically compare sets, using bijections as the primary tool. 

An essential mental construct that Cantor used was his diagnoal argument, which
is often presented when proving the real numbers is not countable. Yet the
central philosophical reasoning behind this argument applies to general sets
which we don't know are countable or not, and this leads to this important
theorem, whose proof I present because it's short and raises questions about
many things.
\begin{CT}
There is no surjection from any set $S$ to the set of all of its subsets
$\mathcal{P}(S)$. In Cantor's words, the cardinal number of $\mathcal{P}(S)$ is
always larger than that of $S$.
\end{CT}
\begin{proof}
When $S$ is empty, it is trival that there's no surjection between $\emptyset$
and $\{\emptyset\}$. Then we consider when $S$ is not empty.

Suppose for now there exists a bijection $f : S \to
\mathcal{P}(S)$. $f$'s being a bijection enables us to apply Cantor's
diagonal argument, because now we can make sure the number of rows and columns
are the same: arrange such a matrix that each column corresponds to an element
$s \in S$, and each row is arranged to correspond to the sequence $f(s)$ for
all $s$, in the same order as the columns. Along the diagonal, flip each
element to define this subset $T \subset S$ such that $s \in T$ iff $s \notin
f(s)$. Although we used the bijection fact to get to this definition, the
definition itself does not require a $f$'s being a bijection; it can be any
function $f : S \to \mathcal{P}(S)$. But we will need it to be a surjection for
the argument that follows. So we now demote $f$ to be any surjection.

Because $f$ is a surjection, there is such $s_0$ that $f(s_0) = T$. Now we ask
if $s_0 \in T$. Note that here we implicitly assume either $s_0 \in T$ or $s_0
\notin T$. In either case, by definition of $T$, we will arrive at a
contradiction. Thus, by \emph{reductio ad absurdum}, it is not the case
that there exists such a surjection $f$.
\end{proof}

Cantor and Russell arrived at two similar paradoxes, both involving the set of
all sets.
\begin{description}
	\item[Cantor's paradox] Consider the set of all sets $\Omega$ and its
		power set $\mathcal{P}(\Omega)$. By Cantor's theorem, the latter should
		be larger than the former. On the other hand, $\mathcal{P}(\Omega)$,
		being a set of sets, should be a subset of $\Omega$, a contradiction.
	\item [Russell's paradox] When Russell was later analyzing Cantor's
		theorem, he wondered what if $S$ itself was $\Omega$. Under the
		assumption that all objects are sets, we thus have $S =
		\mathcal{P}(S)$, and the simple $f: s \mapsto s$ serves as a bijection
		between them. In that case, Cantor's diagnoal subset $T$ becomes the
		famous form that we know of: $T := \{s \mid s \notin s \}$.
\end{description}

For Russell, his paradox was discovered at a bad time, when he was on his way
of finishing his \emph{Principles of Mathematics}, and when he assumed that
there was a universal set. He had hoped for a single, unified solution to the
paradoxes of logic. But to his discouragement, after a few years of various
attempts to fix them in the way he hoped, he was eventually forced to come to
his ramified theory of types, using a hierachy of types to classify
propositions \cite{companion.to.russell.tt}.

Perhaps one reason why the paradoxes were so difficult to fix was that Cantor's
diagonal method had revealed something deeply wrong about the previous
mathematical reasoning, one which goes deeper than even the prominent work of
axiomatic set theory which tried to eliminate these paradoxes. Previously,
mathematicians thought that they were working with real and complete math
objects such as sets and numbers and that they were speaking truths of them as
theorems. Since that had lead to contradictions, there must have been something
wrong in the foundation of what we believed these objects to be or our logic.
Even if we discard the previous foundation entirely and adopt the axiomatic
theory as the new foundation, there must still be the matter of truth and
meaning --- we cannot adopt the axiomatic set theory entirely formally as the
new foundation, without a meaning about what we think is true or not
\cite[Ch.~12]{kleeneitmm}.

\emph{Unfinioned.}

\part{Additional information}

\section{My interest in teaching and assistantship}
Although I have not had any experience as a true teaching assistant, I have
always been attracted to the prospect of teaching others with my knowledge. In
fact, when I am learning something exciting, I will often imagine presenting it
to others and be excited about it if I believe I found a nice way to do that.
(You know, I believe Section~\ref{sec.low.level} and \ref{sec.crypto} are such
examples. Why not take a look if you have not already?)
Similarly, when I am writing something that I want others to see, like this
document, I often take the position of the reader and wonder if she will be
touched by my writing and if she can get my ideas clearly.

To talk more specifically, I am particularly fond of Prof.~Winston's teaching
style, which I discovered when I started self-learning his 6.034 at MIT
\cite{winston.ai}. One important thing I learned from him is that he
demonstrated passion consistently during his teaching, whereas some other
professors I experienced have chosen to present things more objectively.  That
comparison made a huge difference in my perspective --- I feel more engaged and
inspired by his teaching style. I also went on to listen to his \emph{How to
Speak} lecture \cite{winston.htspk} after finishing learning his course.

\section{My career plan}
I plan to enter the academia after my PhD and become a professor somewhere
there is enough freedom of academic expression. The condition is important
because I come from a place where politics and ideology alignment is the
priority, and I see firsthand how damaging that is.

My plan to become a professor is a result of these three things:
\begin{enumerate}
\item My strong academic intersets and my passion in teaching as described
	early.
\item My wish to spread knowledge, advancing my own as well as the others'
	understanding of the subjects I study.
\item Finally, my personal commitment to live and work ethically in a way that
	does not require freedom to come at someone else’s expense. It may be
	easier to gain freedom by aligning with exploitative structures, but I
	cannot accept that path. I hope to take a different approach: to find and
	strengthen spaces within academia where knowledge can be shared openly,
	collaboration can be fair, and students can pursue ideas without fear or
	compromise. Though I recognize that academia is not perfect and  can be
	exploitative (to a less degree), I believe it still offers the best
	opportunity to practice and promote a more just environment in which I work
	and devote my passion.
\end{enumerate}


\bibliographystyle{acm}
\bibliography{main.bib}

\end{document}
