\documentclass{article}

\usepackage{hyperref}

\begin{document}
\section{Security: Introduction into my interests}
I believe security is a good entrance into my current interests. It awards both
attention to detail and high-level systematic view; its subfields connects
directly with both theoretical and practical areas that fascinates me, and it
offers effective tools to solve the imminent problems I face in real life. 

The term \emph{security} in computer science generally carries the sense of
achieving \emph{security goals} through \emph{mechanisms or properties} of a
system, despite the presence of adversaries in a \emph{threat model}. The
threat model has to be carefully chosen and reviewed; there would be no
security against an omnipotent adversary.

Unlike in math, one cannot arbitrarily wiggle the threat model to one's desire.
On the contrary, one's rights are increasingly threatened by the
powerful digitally \cite{eu.digital.1, eu.digital.2, internet.shutdown.2024}. A
prominent example is the continuous push for exceptional access to to people's
data and communication in various forms by governments\footnote{ For instance,
	EU revived the \href{
https://eur-lex.europa.eu/legal-content/EN/TXT/?uri=COM\%3A2022\%3A209\%3AFIN}
{Chat Control} proposal in 2025. See \url{
https://eutechloop.com/time-is-running-chat-control/}.}, ignoring expert
opinions against such measures \cite{keys.under.doormats, bugs.in.our.pockets,
chatcontrolchildprotection}.
I unfortunately has suffered from more aggressive attacks, primarily
systematic, comprehensive, and far-reaching censorship \cite{internet.coup}
\cite[Sect.~5]{chall.censor.circum}, because of my background.

How my security goal of being able to exercise my human rights in the presence of
these powerful adversaries drives my interests will be explained by the
following subsections, each of which also describes how my purely academic
pursuits are intertwined with this realistic drive. Finally, the last
subsection, \ref{sec.crypto}, will transit into the other areas I study.

\subsection{Secure Systems: Understanding the mechanisms} 
\label{sec.secure.systems}
\subsubsection{A high-level view of the Internet}
A high-level understanding of how the digital infrastructures work is essential
to understand what enables these threats. The current networking infrastructure,
in my opinion, has conflicting properties. On one hand, the fundamental problem
of the impossibility to link every two computers requires sharing of links, a
solution that welcomes centralization. On the other hand, the poor scalability
of simple sharing schemes of a link (usually via a switch) necessitates a
better scheme to extend a small network globally. The Internet relies on
topological divisions in its address, and delegates most routing task to each
individual networks (e.g.\ ISPs). Its BGP, opearing in between them, is mainly
concerned exchanging reachability information, whereas its IGPs handle
routing paths and allow different networks to implement different routing
policies.

Therefore, the Internet has become a mixture of centralization and
decentralization, where each end node is managed by an ISP network yet no
single ISP runs the whole Internet . Perhaps surprisingly, I found this hybrid
structure more optimal for localized sabotage at the state level, creating
``sub-Internet''s, each of which is crippled at a different level. 

Unfortunately, the lower layers of the Internet have proven to have a great
inertia for change.  Handley gave a nice discussion on it in 2006
\cite{why.internet.just.works}, and the trend he described has mostly been the
same since then: ``\emph{the core Internet protocols have not changed
significantly in more than a decade, in spite of exponential growth in the
number of Internet users and the speed of the fastest links.}''
\cite{why.internet.just.works}. On the other hand, the protocols in the higher
layers evolved to a much greater degree. As the transport layer welcomes
QUIC\cite{quic}, the application layer has accumulated an enormous amount of
innovation and progress, in particular, onion routing and
Tor\cite{onion.routing, tor}, Bitcoin\cite{bitcoin}, VPN
protocols\cite{openvpn, wireguard} and decentralized instant chat\cite{matrix,
tox}, that fight for digital rights and/or promote decentralization. Yet one
must not overlook the crypto-constructs that lay the foundation for all of
them, which I discuss in detail in Section~\ref{sec.crypto}.

Unfortunately again, because all of these are built on the Internet, a
state-level censor could easily abuse its local authority to target them. A few
countries are notorious to have blocked a vast amount of them to various
extents, employing complicated passive analysis and active probing techniques
\cite{censor.block.1, censor.block.2,censor.block.3, censor.block.4,
censor.block.5,censor.block.6},
Although a state would need to consider the collatoral damage, a totalitarian
scheme would not hesitate to block an entire protocol before it can figure out
how to block it selectively \cite{selective.block.1, censor.block.4} 
Apart from state-level actors, because commerical local ISPs additionally
have an incentive in income and profit, not only do they perform surveillance
and censorship like state-actors \cite{isp.statelike.actions.1,
isp.statelike.actions.2}, they also implement unjust policies easily with their
local control of the infrastructure, like unfairly limiting the use of certain
P2P protocols \cite{isp.block.p2p.1, isp.block.p2p.2, isp.block.p2p.3,
isp.statelike.actions.1}.  Although ISPs argue that these P2P protocols can
take up too much bandwidth, the other side of the story is that ISPs often
oversubscribe and fraudulently advertise the bandwidth of Internet service they
provide \cite{isp.oversubscribe.1, isp.oversubscribe.2}. This essentially is a
probabilistic exploit on its customers --- when almost all of the users happen
to use the maximal bandwidth the ISP sells to them, cogestion and throttling
occur, and the users, not the ISP, ultimately pay the price --- this is also
the same kind of injustice imposed by airlines who oversell tickets, about
which I will not discuss further.

The question of how to build protocols and systems that preserves one's rights
on top of the Internet that powerful adversaries control, fighting against
surveillance, censorship, and overall other unfair practices which the
infrastructures of the current networks happen to enable, is central to my
interest in secure distributed systems.

\subsubsection{Low-level system details}
Beside the threat model, when we talk about security, we also implicitly assume
another model, environment, or host, that encapsulates the problem. For
instance, in the context of isolation, 
\begin{itemize}
	\item The host that encapsulates process isolation is the operating system
		kernel.
	\item The host that encapsulates virtual machine isolation is the
		hypervisor, often including hardware support.
	\item The host that encapsulates air-gapped machine isolation is the
		physical world, or more abstractly, the physical laws.
\end{itemize}
Moreover, these hosts also form a hierachy; the physical world, as the ultimate
host, contains all the others directly or indirectly. Take as example processes
running in an OS from a virtual machine, physically on an air-gapped machine:
in this case, the host OS is contained by the host hypervisor, the hardware,
and finally the physical world.

As a result, a security researcher has to understand these hosts, each to a
different degree, depending on how secure she wants her systems to be. Even if
a system is formally verified to be secure within a host, it can still be
attacked from interactions with an outer host. That is well examplified by
the Meltdown attack, which breaks both kernel and hypervisor's isolation of
memory, because an outer host, the hardware, has a vulnerability
\cite{meltdown}. 

\subsection{Cryptography: A mathematical savior} \label{sec.crypto}
The connection between what could sound like a power struggle and academic
fields lies within (modern) cryptography. For thousands of years since the use
of the earliest symmetric encryption methods like the Ceasar cipher, the
ability to securely communicate through an insecure channel had still been
mostly limited to the privileged who could afford persistent access to physical
secure channels to exchange the keys and reliable safeguarding of the keys.
A stunning turning point was found in what was widely regarded as the beginning
of modern cryptography, Deffie and Hellman's \emph{New Directions in
Cryptography} \cite{new.directions.crypto}, where they gave a practical
mathematical procedure to Merkle's original idea for establishing a key known
only to both parties over an insecure channel.

At first glance, the idea sounded so impossible that Merkel himself faced 
rejections when presenting the idea to his then professor Hoffman and to the
CACM \cite{merkle.rejection}:
\begin{quotation}
	``I am sorry to have to inform you that the paper is not in the main stream
	of present cryptography thinking and I would not recommend that it be
	published in the Communications of the ACM.''

	``Experience shows that it is extremely dangerous to transmit key
	information in the clear.''\cite{merkle.rejection}
\end{quotation}
The rejections represented a concensus of the old cryptography community that
even Shannon concurred with: ``\emph{The key must be transmitted by
non-interceptible means from transmitting to receiving points}''
\cite[p.~670]{shannon.theory.secrecy}.

Such ``paradoxical'' ideas of modern cryptography are what attract me most to
it, because they are the enabler that underlies those systems mentioned in
section~\ref{sec.secure.systems}: TOR, Bitcoin, OpenVPN, Matrix, etc., and the
catalyst that leads to the massive adoption of society advancements such as
e-commerce. Remarkably, in less than 50 years since 1976, we already have a
number of such discoveries in addition:
\begin{description}
\item[Pseudorandom functions] The construction of deterministic algorithms
	whose outputs look like that of a random oracle, by block ciphers like
	AES\cite{aes} or other ways \cite{pseudo.rand.cons.2}, support various
	other essential constructs like one-way functions.

\item[Zero-knowledge proofs] By interactively leveraging challenges that only a
	true prover could easily solve, zero-knowledge proof \cite{zero.knowledge}
	enables one to verify that the prover knows a witness of a problem in NP
	\cite{zero.knowledge.np} without revealing the witness.

\item[Homomorphic encryptions] By 
\end{description}

\bibliographystyle{acm}
\bibliography{main.bib}

\end{document}
