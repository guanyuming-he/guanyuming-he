\documentclass{article}

\usepackage{hyperref}

\begin{document}
\section{Security: Introduction into my interests}
I believe security is a good entrance into my current interests. It awards both
attention to detail and high-level systematic view; its subfields connects
directly with both theoretical and practical areas that fascinates me, and it
offers effective tools to solve the imminent problems I face in real life. 

The term \emph{security} in computer science generally carries the sense of
achieving \emph{security goals} through \emph{mechanisms or properties} of a
system, despite the presence of adversaries in a \emph{threat model}. The
threat model has to be carefully chosen and reviewed; there would be no
security against an omnipotent adversary.

Unlike in math, one cannot arbitrarily wiggle the threat model to one's desire.
On the contrary, one's rights are increasingly threatened by the
powerful digitally \cite{eu.digital.1, eu.digital.2, internet.shutdown.2024}. A
prominent example is the continuous push for exceptional access to to people's
data and communication in various forms by governments\footnote{ For instance,
	EU revived the \href{
https://eur-lex.europa.eu/legal-content/EN/TXT/?uri=COM\%3A2022\%3A209\%3AFIN}
{Chat Control} proposal in 2025. See \url{
https://eutechloop.com/time-is-running-chat-control/}.}, ignoring expert
opinions against such measures \cite{keys.under.doormats, bugs.in.our.pockets}.
I unfortunately has suffered from more aggressive attacks, primarily
systematic, comprehensive, and far-reaching censorship \cite{internet.coup}
\cite[Sect.~5]{chall.censor.circum}, because of my background.

How my security goal of being able to exercise my human rights in the presence of
these powerful adversaries drives my interests will be explained by the
following subsections, each of which also describes how my purely academic
pursuits are intertwined with this realistic drive. Finally, the last
subsection, \ref{sec.crypto}, will transit into the other areas I study.

\subsection{Secure Systems: Understanding the mechanisms} \label{sec.secure.systems}
\subsubsection{A high-level view of the Internet}
A high-level understanding of how the digital infrastructures work is essential
to understand what enables these threats. The current networking infrastructure,
in my opinion, has conflicting properties. On one hand, the fundamental problem
of the impossibility to link every two computers requires sharing of links, a
solution that welcomes centralization. On the other hand, the poor scalability
of simple sharing schemes of a link (usually via a switch) necessitates a
better scheme to extend a small network globally. The Internet relies on
topological divisions in its address, and delegates most routing task to each
individual networks (e.g.\ ISPs). Its BGP, opearing in between them, is mainly
concerned exchanging reachability information, whereas its IGPs handle
routing paths and allow different networks to implement different routing
policies.

Therefore, the Internet has become a mixture of centralization and
decentralization, where each end node connects to an ISP network
(centralization) and ISPs cooperate together to form the whole Internet
(decentralization). Perhaps surprisingly, I found this hybrid structure 
more optimal for localized sabotage at the state level, creating
``sub-Internet''s, each of which is crippled at a different level. 

Unfortunately, the Internet has proven to have a great inertia for change.
Handley gives a nice discussion on it in 2006 \cite{why.internet.just.works},
and the trend he described has mostly been the same since then: ``\emph{he core
Internet protocols have not changed significantly in more than a decade, in
spite of exponential growth in the number of Internet users and the speed of
the fastest links.}'' \cite{why.internet.just.works}. On the other hand, the
protocols in the higher layers evolved to a much greater degree. As the
transport layer welcomes QUIC\cite{???}, the application layer have accumulated
an enormous amount of innovation and progress, in particular, TOR\cite{???},
Bitcoin\cite{???}, OpenVPN\cite{???}, Matrix\cite{???}, that fight for digital
rights and promote decentralization.

Unfortunately again, because all of these are built on the Internet, a
state-level censor could easily abuse its local authority to target specific
protocols. China is notorious to have blocked a vast amount of protocols, and
it temporarily blocked QUIC until it figured out how to block only specific
domains. The question of how to build protocols and systems on top of the
Internet that adversaries control, is central to my interest in secure
systems.

\subsubsection{Low-level system details}

\subsection{Cryptography: A mathematical savior} \label{sec.crypto}
The connection between what could sound like a power struggle and academic
fields lies within (modern) cryptography. For thousands of years since the use
of the earliest symmetric encryption methods like the Ceasar cipher, the
ability to securely communicate through an insecure channel had still been
mostly limited to the privileged who could afford persistent access to physical
secure channels to exchange the keys and reliable safeguarding of the keys.
A stunning turning point was found in what was widely regarded as the beginning
of modern cryptography, Deffie and Hellman's \emph{New Directions in
Cryptography} \cite{new.directions.crypto}, where they gave a practical
mathematical procedure to Merkle's original idea for establishing a key known
only to both parties over an insecure channel.

At first glance, the idea sounded so impossible that Merkel himself faced 
rejections when presenting the idea to his then professor Hoffman and to the
CACM \cite{merkle.rejection}:
\begin{quotation}
	``I am sorry to have to inform you that the paper is not in the main stream
	of present cryptography thinking and I would not recommend that it be
	published in the Communications of the ACM.''

	``Experience shows that it is extremely dangerous to transmit key
	information in the clear.''\cite{merkle.rejection}
\end{quotation}
The rejections represented a concensus of the old cryptography community that
even Shannon concurred with: ``\emph{The key must be transmitted by
non-interceptible means from transmitting to receiving points}''
\cite[p.~670]{shannon.theory.secrecy}.

Such ``paradoxical'' ideas of modern cryptography are what attract me most to
it, because they are the enabler that underlies those systems mentioned in
section~\ref{sec.secure.systems}: TOR, Bitcoin, OpenVPN, Matrix, etc., and the
catalyst that leads to the massive adoption of society advancements like
e-commerce. Remarkably, in less than 50 years since 1976, we already have a
number of such discoveries in addition:
\begin{description}
\item[Pseudorandom functions] The construction of deterministic algorithms
	whose outputs look like that of a random oracle, by block ciphers like
	AES\cite{aes} or other ways \cite{pseudo.rand.cons.2}, support various
	other essential constructs like one-way functions.

\item[Zero-knowledge proofs] By interactively leveraging challenges that only a
	true prover could easily solve, zero-knowledge proof \cite{zero.knowledge}
	enables one to verify that the prover knows a witness of a problem in NP
	\cite{zero.knowledge.np} without revealing the witness.

\item[Homomorphic encryptions] By 
\end{description}

\bibliographystyle{acm}
\bibliography{main.bib}

\end{document}
